\section{Categorization of methods}
\label{sec:categorization}

\begin{quotation}%
Notoriamente no hay clasificación del universo que no sea arbitraria y 
conjetural. La razón es muy simple: no sabemos qué cosa es el universo.
\\\quotationsource \Person[Jorge Luis]{Borges} (\citeyear{Borges1952})
\end{quotation}

\noindent The following categorization of data structuring methods is a result
of the collection of methods analyzed in chapter~\ref{ch:methods}. Given these
methods one can categorize them by history and origin, by type of application,
by complexity, and by many other criteria.  This approach, however, can result
in rather arbitrary classifications, because a single facet has to be chosen
and because most facets are not selective for all instances.  Another approach,
that better fits to how people cognitively perceive and classify things, is
grouping based on prototypes which act as cognitive reference points
\cite{Lakoff1987,Rosch1983}.  Following this approach, categories of data
structuring methods are not defined by selected features, but data structuring
methods are clustered by similarity, until prototypical methods emerge. A
\Term{prototype} can act as good example of a certain category, while other
instances belonging to this category are less central. For instance in western
society, a chair is a central prototype of furniture, although other furniture
may share little properties with chairs. 

\begin{table}
\centering
\begin{tabular}{|l|l|l|}
\hline
\textbf{category} & \textbf{main purpose} & \textbf{examples} \\
\hline
encodings       & express data & \term{Unicode}, \term{Base64} \\
storage systems & store data & \acro{NTFS}, \acro{RDBMS} \\
identifier and query languages   & refer to data  & \acro{URI}, \acro{XPath} \\
structuring and markup languages & structure data & \acro{XML}, \acro{CSV}, \acro{RDF} \\
schema languages  & constrain data & \acro{BNF}, \acro{XSD} \\
conceptual models & describe data  & \term{Mind Maps}, \acro{ERM} \\
\hline
\end{tabular}
\caption{Prototype categorization of data structuring methods}
\label{tab:protocat}
\end{table}

The final prototype categorization of data structuring methods found in this
study is summarized in table~\ref{tab:protocat}. The categorization is a novel
result, because comparative studies of data structuring methods, as broad as
given in this thesis, have not been conducted yet. The prototypes have partly
been anticipated in the division of chapter~\ref{ch:methods}:

\begin{itemize}
 \item encodings: section 
  \ref{sec:characters}
 \item storage systems: section 
  \ref{sec:filesystems} and \ref{sec:databases}
 \item identifier and query languages: section
   \ref{sec:identifiers} and \ref{sec:queries}
 \item structuring and markup languages: section
  \ref{sec:dsl} and \ref{sec:markuplanguages}
 \item schema languages: 
   section \ref{sec:schemas}
 \item conceptual models:
   section \ref{sec:modelangs} and \ref{sec:diagrams}
\end{itemize}

To further validate this result, the prototype categorization was analyzed to
find a supporting facet that best divides categories by one aspect of data
structuring. As the \term{research question} asks for general methods that
span a wide range of digital technologies, the supporting facet should be
independent from particular use cases and applications. It was found that the
{\em main purpose} of a method can be used as dividing facet. This purpose
describes what a method mainly does with data. For instance the main purpose of
storage systems like databases and file systems is storage.  The data
structuring method's main purpose can act as guideline to find the nearest
prototype from table~\ref{tab:protocat}. Still this dividing facet should not
be confused with a strict classifier as known from more formal approaches of
categorization. Instances from each category can also serve multiple purposes,
just like one can use a chair to stand on it when changing a light bulb.


