%\vfill
%\pagebreak
\section{Basic patterns}
\label{sec:basic-patterns}

Basic patterns highlight the the most fundamental strategies of describing and
structuring data elements. The patterns can be found anywhere at single data
elements. The patterns \pattern{label} and \pattern{atomicity} take an element
as such without further inspection. The patterns \pattern{size},
\pattern{optionality}, and \pattern{prohibition} also include the idea of
content which a data element is build of and which can be shaped in a specific
way for each of the patterns.

\input{datapatterns/basic}

\clearpage
\section{Combining patterns}
\label{sec:combining-patterns}

The primarily purpose of combining patterns is to connect multiple data
elements to larger structures. This combination can be done by several methods.
Combining patterns include \pattern{sequence} and \pattern{graph} which
structure multiple elements on the same level, and \pattern{container},
\pattern{dependence}, and \pattern{embedding} which include an idea of
subsuming elements.  As shown in figure~\ref{fig:compatterns}, combining
patterns are hierarchically connected to each other by general implications and
by the context they occur in (both connections shown by arrows).

The most fundamental and most abstract patterns are \pattern{embedding} and
\pattern{dependence}. More visible data combinations can be found as general
collections (\pattern{container}), such as sets of files and records, and in
form of ordered data (\pattern{sequence}) and graph structures
(\pattern{graph}). Together with these combining patterns one often finds two
basic patterns and two continuing patterns, \pattern{size} and
\pattern{atomicity}, which reflect a number of (possibly combined) elements and
the indivisible items to be connected, respectively. The continuing patterns
\pattern{separator} and \pattern{etcetera} are needed to indicate borders and
connections between elements and to indicate that a combination is incomplete.


\vfill
\begin{figure}[h]
\centering
\begin{tikzpicture}
\matrix[every node/.style={draw,ellipse,},column sep=3mm,row sep=5mm,
 ] {
  & & & \node (separator) {\pattern{separator}}; \\
  & \node (graph) {\textbf{\pattern{graph}}}; &
	\node (sequence) {\textbf{\pattern{sequence}}}; &
	\node (etcetera) {\pattern{etcetera}}; \\
  & \node (atomicity) {\pattern{atomicity}}; &
	\node (container) {\textbf{\pattern{container}}}; & \\
	\node (dependence) {\textbf{\pattern{dependence}}}; &
  \node (size) {\pattern{size}}; &
  \node (embedding) {\textbf{\pattern{embedding}}}; & \\
};
\draw[garc] (graph) to (container);
\draw[garc] (graph) to[out=-160,in=90] (dependence);
\draw[garc] (atomicity) to (size);
\draw[garc] (container) to (size);
\draw[garc] (container) to (embedding);
\draw[garc] (etcetera) to[out=-100,in=30] (embedding);
\draw[garc] (etcetera) to (container);
\draw[garc] (sequence) to (container);
\draw[garc] (separator) to[out=-20,in=0] (embedding);
\draw[garc] (separator) to (sequence);
\end{tikzpicture}
\caption{Connections between patterns (combining patterns in bold)}
\label{fig:compatterns}
\end{figure}
\vfill


\clearpage
\input{datapatterns/combining}

\clearpage
\section{Relationing patterns}
\label{sec:relationing-patterns} 

The following data patterns primarily relate elements to each other. In
contrast to combining patterns (section~\ref{sec:combining-patterns}), which
primarily group data to larger structures, the connections established by
relationing patterns are more between data elements which serve different
purposes. Each pattern solves a general problem in data structuring and
description, for instance complexity (solved by \pattern{encoding}) and
redundancy (solved by \pattern{normalization} and \pattern{derivation}).
Figure~\ref{fig:relpatterns} groups relationing patterns by connections of
implication or context in three levels: the most fundamental patterns include
\pattern{identifier} and \pattern{derivation}. Based on these patterns one can
find instances of \pattern{encoding} and \pattern{flag}, among other
structures. Finally instances of \pattern{normalization} and \pattern{schema}
are based on patterns of the second level. Figure~\ref{fig:relpatterns}
includes more patterns connected to relational patterns by informal implication
and context. A full diagram of connections is given in appendix
\ref{appendixC}.

\vfill
\begin{figure}[h]
\centering
\begin{tikzpicture}
\matrix[every node/.style={draw,ellipse,},column sep=3mm,row sep=5mm,
 ] {
 \node (optionality) 
 	{\pattern{optionality}}; &
 \node (garbage) 
 	{\pattern{garbage}}; &
 \node (prohibition) 
 	{\pattern{prohibition}}; \\
 \node (normalization) 
 	{\pattern{normalization}}; &
 \node (schema) 
 	{\textbf{\pattern{schema}}}; \\
 \node (encoding) 
 	{\textbf{\pattern{encoding}}}; &
 \node (flag) 
 	{\textbf{\pattern{flag}}}; &
 \node (void) 
 	{\pattern{void}}; \\
 \node (identifier) 
 	{\textbf{\pattern{identifier}}}; &
 \node (derivation) 
 	{\textbf{\pattern{derivation}}}; \\
}; % end of matrix
\draw[garc] (optionality) to (schema);
\draw[garc] (garbage) to (schema);
\draw[garc] (prohibition) to (schema);
\draw[garc] (normalization) to (encoding);
\draw[garc] (schema) to (flag);
\draw[garc] (flag) to (derivation);
\draw[garc] (encoding) to (identifier);
\draw[garc] (encoding) to (derivation);
\draw[garc] (void) to (derivation);
%\draw[garc,dashed] (flag) to (void);
%\draw[garc,dashed] (garbage) to (identifier);
\end{tikzpicture}
\caption{Connections between patterns (relationing patterns in bold)}
\label{fig:relpatterns}
\end{figure}
\vfill

\clearpage
\input{datapatterns/relational}

\clearpage
\section{Continuing patterns}
\label{sec:continuing-patterns}

Continuing patterns are easily overlooked because they don't show explicit
encoding. Instead these patterns primarily refer to data that is continued
elsewhere, possibly even at another level or another realm of description. The
most prominent continuing pattern is the \pattern{separator} which indicates a
border between data elements. The ``punctuation of data'' is often visible in
form of brackets, delimiters, and whitespace. The \pattern{etcetera} pattern is
less welcome because it shows that data is rarely complete. Continuation
markers such as `\verb|et al.|' still have their use because gaps and limits
would be hidden without them. In the end all data refers to something in the
realm of reality which is never fully encoded in data. The \pattern{garbage}
pattern can indicate missing data as well but in this case there is nothing
more to be encoded. Garbage values such as `\verb|n/a|` and `\verb|NULL|' make
irrelevant or inapplicable values explicit instead of just omitting them.
Omission on the other hand is the basic idea of the \pattern{void} pattern.  If
patterns would be given more colloquial nicknames, the continuing patterns
could also be named ``the glue`` (\pattern{separator}), ``the hint''
(\pattern{etcetera}), ``the ugly'' (\pattern{garbage}), and ``the mystery''
(\pattern{void}).

%\clearpage
\input{datapatterns/continuing}
\clearpage
