\section{Patterns and pattern languages}
\label{sec:patterntheory}

\begin{quotation}%
Design and programming are human activities; forget that and all is lost.
\\\quotationsource \Person[Bjarne]{Stroustrup} (\citeyear{Stroustrup1997})
%  The C++ Programming Language. pp. 693.
\end{quotation}

\noindent The novel approach of this thesis is to use \Term[pattern]{patterns}
for data description, independent from particular structuring methods and
technologies. This section will first introduce the notion of patterns, then
summarize existing works that deal with patterns in data structuring and
finally give an example.

Patterns as systematic tools for describing good design practice were first
introduced by \Person[Christopher]{Alexander}, \Person[Sarah]{Ishikawa}, and
\Person[Murray]{Silverstein} \citeyear{Alexander1977}. They identified 253
existing architectural patterns from entire regions and cities to buildings,
rooms, and furniture. In Alexander's original definition \citeyear[p.
x]{Alexander1977} ``each pattern describes a problem which occurs over and over
again in our environment, and then describes the core of the solution to that
problem, in such a way that you can use this solution a million times over,
without ever doing it the same way twice.'' Patterns can be found by observing
current practice and then looking for commonalities in solutions to a problem.
In contrast to simple rules or best practice guidelines, a pattern, however,
does not solve the problem by providing a particular solution, but by showing
benefits and consequences. Each pattern provides a solution and each solution
has some tradeoffs. The pattern description guides designers in their decisions
of particular solutions for particular applications. Each pattern is given a
name, which can be used to refer to one pattern from another. The full
potential of patterns unfolds if a set of patterns is collected and combined in
a \Term{pattern language}. In Alexander's words ``a pattern language is a
network of patterns that call upon one another. Patterns help us remember
insights and knowledge about design and can be used in combination to create
solutions.`` A pattern language for writing patterns was presented by
\textcite{Meszaros1997}.

The pattern language approach with its application in architecture has been
adopted in other fields of engineering, especially in \term{software
engineering} \cite{Beck1987}.  \Person[Erich]{Gamma}, \Person[Richard]{Helm},
\Person[Ralph]{Johnson}, and \Person[John]{Vlissides} (the so-called `gang of
four') published an influental book on \Term{design patterns} in object
oriented programming \cite{Gamma1994}.  In 1995 \Person[Ward]{Cunningham}
created the Portland Pattern Repository \cite{Cunnigham1995}, accompanied by
WikiWikiWeb, which was the world's first \term{wiki}.\footnote{The Portland
Pattern Repository and WikiWikiWeb are still active at \url{http://c2.com/}.}

Although these design patterns are used for the creation of computer programs,
they do not reflect problems and solutions of data structuring as analyzed in
this thesis. Design patterns refer to dynamic processes, while digital documents
are static.  General patterns in description and structuring of data must also
be separated from \Term{pattern recognition}, as practiced in \term{data
mining} and other statistical methods of \term{machine learning}. These
quantitative methods can only recognize structures within the boundaries of a
fixed method of data description (for instance statistical patterns in lists of
numbers without questioning the nature of numbers and lists). A general
limitation of existing approaches is the focus to one specific formalization
method. This practical limitation blocks the view to more general data
patterns, independent from a particular encoding, and it conceals blind spots
and weaknesses of a chosen formalism. Some works about patterns in particular
data description languages have been mentioned in
section~\ref{sec:relatedworks}.

\subsectionexample{One data element, many patterns}

The following example shall illustrate the application of patterns in data
description. The patterns mentioned here anticipate members of the final
pattern language summarized in chapter~\ref{ch:patterns}. A more complex
example is given in appendix~\ref{appendixD}.

Given the following sequence of twelve bytes: 

\begin{center}
\verb|44 75 62 6c 69 6e 2c 20 4f 68 69 6f|\\
\end{center}

\noindent How can this particular piece of data be structured and described? To
start with, we need at least some context or indication. Let's assume each byte
corresponds to one character. This kind of correspondence can be summarizes as
\pattern{encoding} pattern. Given ASCII or Unicode encoding, the sequence is:

\begin{center}
\verb|Dublin, Ohio|
\end{center}

\noindent Several patterns provide obvious solutions to further description:

\begin{itemize}

	\item The data may be a list of two elements, \texttt{Dublin} 
		  and \texttt{Ohio}	(\pattern{sequence} pattern).

	\item It may consist of two elements as part of an unsorted collection
		  (\pattern{container} pattern), so
	      \texttt{Ohio, Dublin} should be equal to \texttt{Dublin, Ohio}.

	\item It may just refer to the name ``Dublin, Ohio'' without any 
		  relevant structure (\pattern{label} pattern).

    \item It may consist of two words, one of which (\texttt{Ohio}) being 
		  used as qualifier for the other (\texttt{Dublin}).

\end{itemize}

\noindent Given the last interpretation, a \emph{qualifier} may be a pattern in
its own right or it may be an example of a more general \pattern{flag} pattern
to indicate the interpretation of one element (\texttt{Dublin}) by another
(\texttt{Ohio}).

One can further deconstruct the structure of a data element to parts, a typical
process of description (\pattern{schema} pattern):

\begin{center}
$\underbrace{\texttt{Dublin}}_{1}%
\underbrace{\texttt{, }}_{2}%
\underbrace{\texttt{Ohio}}_{3}$
\end{center}

\noindent The third part is attached as additional element to the first 
(\pattern{dependence} pattern), and it may unambiguously refer into a registry of
allowed qualification values (\pattern{identifier} pattern). The second part
acts as connection or delimiter (\pattern{separator} pattern). Even its two
bytes (\texttt{, }) can have structure: the whitespace character is often
used as filling without significance (\pattern{garbage} pattern).

In summary one can identify several typical structuring methods in the twelve
bytes given above. The interpretation, however, does not need to be right ---
depending on context the sequence could mean virtually anything --- but
patterns help to reveal interpretations that were most likely intended when
creating the data.

