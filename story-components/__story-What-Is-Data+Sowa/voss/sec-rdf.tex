\subsection{RDF}
\label{sec:rdf}

The \Tacro{Resource Description Framework}{RDF} dates back to the
\tacro{Meta Content Framework}{MFC} which \Person[Ramanathan]{Guha}
had created in the 1990s at \term{Apple} \cite{Andreessen1999,Guha1996}.
\textcite{Guha1997} submitted the idea of \acro{MFC} to the \acro{W3C}, where 
it evolved to a general graph-based metadata framework, first released as 
\acro{RDF} specified by \textcite{Lassila1999}. Merged with ideas of 
\textcite{TBL1997}, the focus had widened to metadata about any objects for
creating a ``\term{Semantic Web}'', that can express knowledge
about the world itself \cite{BernersLee2001a}. The ambitious aim of \acro{RDF}
can be traced back further to the artificial intelligence project \term{Cyc},
which \person[Ramanathan]{Guha} was co-leader of in 1987--1994, and to
the original proposal of the \acro{WWW} \cite{BernersLee1989}.
We will first describe the basic components of \acro{RDF}, show how its 
structure can be described and extended, and list several serialization forms.
Afterwards we will discuss how semantics is brought to \acro{RDF} via an
algebra over RDF graphs.

\subsubsection{RDF components}

In its most basic form --- that is without additional techniques such as 
\acro{RDFS}, \acro{OWL}, \acro{SKOS}, etc. --- \acro{RDF} is just a, graph-based 
data structuring language. The \acro{RDF} data model is defined as abstract 
syntax by \textcite{Klyne2004} as follows: an \Term[graph!in RDF]{RDF graph} 
is a set of \Term[triple (RDF)]{RDF triples} (also known as \Termstyle{statement}s)
each consisting of a \Term[subject (RDF)]{subject}, a \Term[predicate (RDF)]{predicate}
(also known as \Term[property (RDF)]{property}), and an \Term[object (RDF)]{object}.
The nodes of an RDF graph are its subjects and objects, and the edges are labeled by
its predicates. Because RDF graphs are defined on mathematical sets, each particular 
combination of subject, predicate, and object is only counted once in a graph.
In summary, an RDF graph can be described as multigraph with labeled edges, possible
loops (triples where subject and object coincide), and partly labeled nodes. This 
multigraph may contain two kinds of graph labels: First an \Term{URI reference} 
(or \Termstyle{URIref}) is an absolute, percent-encoded \acro{URI} or \acro{IRI}. 
Two URIrefs are equal if and only if they compare as equal as encoded strings.%
\label{term:uriref}%
\footnote{This definition of equality is not based on normalized \acro{IRI}s, but 
on Unicode character string comparision \cite[6.4]{Klyne2004}. This makes 
\url{http://example.com/\%41} and \url{http://example.com/A} two distinct URIrefs, 
although in most applications, after normalization of percent-encoding, the former 
results in the latter. The ambiguity cannot be solved in general, because there
is no general canonicalization algorithm for all types of \acro{IRI}s. It is 
possible but strongly discouraged to have two different URIrefs that 
percent-encode the same \acro{IRI}.} URIrefs are treated as identifiers
for \Term[resource (RDF)]{RDF resources}, which beside triples are the central 
part of the Resource Description Framework. No assumptions are made about the 
nature of resources, but the same URIref always refers to the same resource
\cite[section 1.2]{Hayes2004}. \acro{RDF} allows different URIrefs to refer to
the same resource (synonyms), but unlike natural langugages it is assumed to
have no homonym URIrefs just by definition.\footnote{By this, \acro{RDF}
can be seen as one of many attempts to create a `perfect language', in which same
words always refer to same objects. The history of other attempts and their
failures have been illustrated vivid by \textcite{Eco1995}.}
It should be noted that RDF graphs cannot contain resources which are not linked
to other resources: you state that some resource plays a specific role in at least
one \acro{RDF} triple, but you cannot state that a selected resource only `exists`.

The second type of graph labels are \Term[literal (RDF)]{RDF literals}. A
literal is a Unicode string, which should be in \term{Normalization Form C}.
Optionally it is combined with either a lowercase \Term{language tag} as
defined by \textcite{RFC4646}, or with a \Term{datatype URI} being an URIref.
Literals with datatype are called \Term{typed literals} in contrast to
\Term{plain literals}. Two literals are equal if they hold the same Unicode
string (also called its \Term[lexical form (RDF literal)]{lexical form}), and
(if given) the same language tag datatype \acro{URI}. Datatypes may enforce
restrictions and normalization rules on lexical forms, but the details of this
rules are out of the scope of basic \acro{RDF}.\footnote{To give an example,
the \term{XML Schema} specification \cite{Biron2004} defines the datatype
\rdf{xs:boolean} with four allowed lexical values (``\texttt{true}'',
``\texttt{1}``, ``\texttt{false}'', ``\texttt{0}'') that map to the canonical
literal values ``\texttt{true}'' and ``\texttt{false}''.}

In addition to labeled nodes, at least subjects and objects can be unlabeled
\Term[blank node]{blank nodes}. Blank nodes are treated as variables for 
unknown URI references in one particular RDF graph. You can state that two
blank nodes in one graph refer to the same resource, but blank nodes from
different graphs are disjoint, unless you replace them with URIrefs. In
practice, blank nodes are identfied by arbitrary identifiers, that are not
shared among different graphs. As laid out by \textcite{Carroll2003}, blank
nodes can make it hard to check, whether two graphs are equal, to calculate 
a canonical representation of an RDF graph, and to remove all infereable
tuples from a graph.\footnote{\label{fn:gi}It is assumed
that the underlying \tacro{graph isomorphism problem}{GI} is strictly harder 
than \tacro{polynomial time}{P}, and strictly easier than nondeterministic 
\tacro{non-deterministic polynomial time}{NP} \cite{Kobler2006}.}

\begin{figure}
\centering
\begin{tikzpicture}[orm,arrow/.style={->,thick},box/.style={draw,rectangle,thick}]
\node[box,inner sep=4pt] (r) {info:oclcnum:318301686};

\node[right=15mm of r,yshift=8mm] (t) {"The C programming language"@en};
\draw[arrow,out=35,in=180] (r) to (t);
\node[left=1mm of t,yshift=2mm] {dc:title};

\node[right=15mm of r] (y) {"1978"\textasciicircum\textasciicircum};
\draw[arrow] (r) to (y);
\node[left=1mm of y,yshift=2mm] {dc:date};
\node[right=0mm of y,box,inner sep=2pt,yshift=-0.5mm] 
    {http://www.w3.org/2001/XMLSchema\#gYear};

\node[draw,thick,circle,right=16mm of r,yshift=-8mm] (b1) {};
\node[draw,thick,circle,right=16mm of r,yshift=-15mm] (b2) {};

\draw[arrow,out=-25,in=180,shorten >=1mm] (r) to (b1);
\node[left=1mm of b1,yshift=3mm] {dc:creator};
\draw[arrow,out=-75,in=180,shorten >=1mm] (r) to (b2);
\node[left=1mm of b2,yshift=2mm] {dc:creator};

\node[right=18mm of b1] (v1) {"Kernighan"};
\draw[arrow] (b1) to node[yshift=2mm] {foaf:name} (v1);
\node[right=18mm of b2] (v2) {"Ritchie"};
\draw[arrow] (b2) to node[yshift=2mm] {foaf:name} (v2);

\matrix[row sep=2mm] (m) at (6,-3.2) {
 \node[box,anchor=west,label=left:{dc:title =},inner sep=2pt]  
     {http://purl.org/dc/elements/1.1/title}; \\
 \node[box,anchor=west,label=left:{dc:creator =},inner sep=2pt] 
    {http://purl.org/dc/elements/1.1/creator}; \\
 \node[box,anchor=west,label=left:{foaf:name =},inner sep=2pt] 
   {http://xmlns.com/foaf/0.1/name}; \\
};
\draw[decorate,decoration={brace},xshift=-2cm] 
  (m.south west) -- node[anchor=east,inner sep=2mm]
  {abbreviations} (m.north west);

\end{tikzpicture}
\caption{Example of a simple RDF graph}
\label{fig:rdfgraphexample}
\end{figure}

Figure \ref{fig:rdfgraphexample} shows a simple \acro{RDF} graph consisting
of six triples. URIrefs are depicted as rectangles, blank nodes as circles,
and predicates are labeled arcs. Literals are shown in quotation marks,
optionally followed by ``\verb|@|'' and a language tag, or by
``\verb|^^|'' and a datatype \acro{URI}. The same graph in
Turtle syntax is shown in figure~\ref{fig:rdfgraphexturtle}.

\begin{table}
\centering
\begin{tabular}{|l|c|c|c|c|c|c|}
\hline
\textbf{\acrostyle{RDF} extension} & \textbf{subject} & \textbf{predicate} & 
\textbf{object} & \textbf{datatype} & \textbf{lang.} & \textbf{graph} \\
\hline
standard          & $U \cup B$        & $U$        & $U \cup B \cup L$ & $U$ & $T$ & -- \\
\hline
symmetric         & $U \cup B \cup L$ & $U$        & $U \cup B \cup L$ & & & \\
\hline
generalized       & $U \cup B \cup L$ & $U\cup B\cup L$ & $U\cup B \cup L$ & & & \\
\hline
full blanks       & & $U\cup B$ & & $U\cup B$ & & $U\cup B$ \\
\hline
named graphs      &  &  &  & & & $U$ \\
\hline
language URIs     & & & & & $U [\cup B]$ & \\
\hline
\end{tabular}

\begin{tabular}{l}
$U$: URIref, $B$: blank node, $L$: Literal,
$S$: Unicode string, $T$: language tag \\
$L= S \cup (S \times U) \cup (S \cup T)$ 
and $U, B, L, S, T$ are pairwise disjoint. \\
The set of blank nodes $B$ is partitioned into disjoint sets
for different \acro{RDF} graphs.
\end{tabular}

\caption{Definitions of the \acrostyle{RDF} data model and its extensions}
\label{tab:rdfvariants}
\end{table}

\subsubsection{The model and its extensions}
\label{sec:rdfmodels}

An \acro{RDF} graph can formally be defined as subset of the set 
$(U \cup B) \times U \times (U \cup B \cup L)$ of all triples, as laid
out in table~\ref{tab:rdfvariants}. You can think of several
useful extensions of the standard \acro{RDF} data model.
First literals can also be allowed as subject of a triple. This 
extension to `symmetric' \acro{RDF} allows reversing the direction of 
any triple by switching subject and object. In standard \acro{RDF} you
cannot state, that a given literal is the `name of' a given resource, but 
only that a resource is `named as' a literal. Symmetric \acro{RDF} is 
allowed at least internally in many \acro{RDF} applications, for instance
in \acro{SPARQL} \cite[sec. 12.1.4.]{Prudhommeaux2008}.
Second you could allow literals and blank nodes at any part of a triple.
This extension to \Term{generalized RDF} is allowed for instance in
\acro{OWL2} \cite[sec. 2.1]{Schneider2009}. An \acro{ORM} model of
generalized RDF is given in figure~\ref{fig:genrdfmodel}.
Third you could allow blank nodes at every place where URIrefs are allowed,
that means also as predicates and/or data types.

A different popular extension
allows labeling whole RDF graphs by URIrefs. This extension was 
introduced as \Term{named graphs} by \textcite{Carroll2005}. It has been 
adopted for instance in \term{triple store}s (in this case also known as 
\term{quadstore}s) that deal with multiple RDF graphs.
Finally the
replacement of language tags by URIrefs or blank nodes would repair 
another design failure of standard \acro{RDF}.\footnote{In standard \acro{RDF}
you cannot refer to language tags in statements because language tags are 
disjoint with URIrefs and blank nodes.}

\subsubsection{Serializations}

The \acro{RDF} data model is not bound to a specific syntax, but there a several
serialization formats. The most common format is \acro{RDF/XML} which uses 
\acro{XML} \cite{Beckett2004}. Suprisingly, some allowed RDF graphs cannot be 
expressed in \acro{RDF/XML} because they contain literals or predicate \acro{URI}s
with Unicode codepoints forbidden in \acro{XML}~1.0. There are also numerous alternative
ways to describe the same RDF graph in \acro{RDF/XML}, which makes it hard 
to use generic \acro{XML} tools to process general \acro{RDF} data.
\Term{TriX} (RDF Triples in XML) is an alternative \acro{XML} based syntax for
\acro{RDF}, that additionally provides for serializing several (named) graphs in
a single document \cite{Carroll2004}. \acro{RDF/JSON} and \acro{JSON-LD} are serializations of
\acro{RDF} in \tacro{JavaScript Object Notation}{JSON}, which is popular
in several scripting languages \cite{Alexander2008,Sporny2012}.
\Term{N-Triples} is a simple, plain text
serialization that was created for test cases \cite{Grant2004}. RDF graphs in
N-Triples are written one triple per line and the character set is 7-bit 
US-ASCII, but still the format is capable of encoding all \acro{RDF}. 
\Term{Turtle} (Terse RDF Triple Language) was created by \Person[David]{Beckett}
as more flexible and readable syntax extension of N-Triples \cite{Beckett2007}.
Turtle is probably the most popular \acro{RDF} serialization format next to 
\acro{RDF/XML}; an example is shown in figure~\ref{fig:rdfgraphexturtle}.
\Term{TriG} syntax \cite{Bizer2007,Carroll2005} extends Turtle by using curly 
brackets to group triples into multiple graphs, and to precede each graph by an
URIref as its name. Apart from minor syntax variants, that can be added 
automatically (an equal sign before and a dot after each graph), TriG is also 
compatible with the syntax of \Tacro{Notation3}{N3},\footnote{However both use
different models of \acro{RDF}: Trig is based on named graphs and N3 on standard
RDF with a custom \rdf{rei:} reification vocabulary.} which is another superset of 
Turtle. \acro{N3} extends Turtle with features such as variables, formulae, logical 
implications, and functional predicates, that can be used to abbreviate common 
URIrefs and patterns of \acro{RDF} statements \cite{BernersLee2008}. A summary 
of syntax elements of \term{Turtle} and \term{Notation3} is given in 
table~\ref{tab:n3syntax}.

\begin{figure}
\centering
\begin{minipage}{9cm}
\begin{lstlisting}[language=turtle]
@prefix xs: <http://www.w3.org/2001/XMLSchema#> .
@prefix dc: <http://purl.org/dc/elements/1.1/> .
@prefix foaf: <http://xmlns.com/foaf/0.1/> .

<info:oclcnum:318301686>
  dc:title "The C programming language"@en ;
  dc:date "1978"^^xs:gYear ;
  dc:creator [ foaf:name "Kernighan" ] ,
             [ foaf:name "Ritchie"   ] .
\end{lstlisting}
\end{minipage}
\caption{Simple RDF graph in Turtle syntax}
\label{fig:rdfgraphexturtle}
\end{figure}

\begin{table}
\centering
\begin{tabular}{|l|l|}
\hline
\textbf{Syntax element(s)} & \textbf{purpose} \\
\hline
\rdf{@prefix} & defines a namespace shortcut to abbreviate URIrefs \\
\rdf{@base}   & defines a standard prefix for URIrefs \\
\rdf{<X>} & URIref with URI \url{X} \\
\rdf{"..."}, \rdf{"""..."""} & literals \\ 
\rdf{"..."@X} & literal with language tag \rdf{X} \\ 
\rdf{"..."\^\^<X>} & literal with datatype URIref \url{X} \\ 
\rdf{.} & marks the end of a statement \\
\rdf{;} & following statement(s) have same subject \\
\rdf{,} & following statement(s) have same subject and predicate \\
\rdf{a} & shortcut for \rdf{rdf:type} \\
\rdf{_:X}, \rdf{[ ]} & blank node with local id \rdf{X} or without specific id \\
numeric literals & \rdf{xs:integer} and \rdf{xs:float} as datatype \\
\rdf{( )} & \rdf{rdf:List}, \rdf{rdf:first}, \rdf{rdf:rest}, and \rdf{rdf:nil}.\\
\rdf{#...} & comment \\
\hline
\rdf{=} & \rdf{owl:sameAs} \\
\rdf{!}, \rdf{^}, \rdf{@forSome} & statements with blank nodes \\
\rdf{=>}, \texttt{<=} & \rdf{log:implies} \\
\rdf{\{  \}} & statements with \rdf{rei:} reification and formulae \\
\rdf{@forAll} & \rdf{rei:universals} \\
\rdf{?x}, \rdf{:y} & variables in formulae \\
\hline
\end{tabular}
\begin{tabular}{l}
 \rdf{rdf:} is a shortcut for \url{http://www.w3.org/1999/02/22-rdf-syntax-ns#} \\
 \rdf{owl:} is a shortcut for \url{http://www.w3.org/2002/07/owl#} \\
 \rdf{xs:} is a shortcut for \url{http://www.w3.org/2001/XMLSchema#} \\
 \rdf{log:} is a shortcut for \url{http://www.w3.org/2000/10/swap/log#} \\
 \rdf{rei:} is a shortcut for \url{http://www.w3.org/2004/06/rei#} \\
\end{tabular}
\caption{Syntax elements of Turtle (above) and Notation3 (additionally below)}
\label{tab:n3syntax}
\end{table}

\subsubsection{Vocabularies}

A common technique used in all syntaxes (except \term{N-Triples}) is the abbreviation
of URIrefs with \term{namespace} prefixes. In practice it is often assumed, that all
resources, under one namespace prefix share same properties and that they belong to 
one common \Term{RDF vocabulary}. An \term{RDF vocabulary} is a set of URIrefs and
statements, that are created, described, and maintained for a specific use-case. 
If the vocabulary makes use of an RDF schema language like \acro{RDFS} or \acro{OWL}
(see section~\ref{sec:rdfschemas}), or if it implies other logical inference rules, 
the vocabulary can also be called an \term{ontology}.

The basic \acro{RDF} data model as described by \textcite{Klyne2004} includes 
only the predefined datatype URIref \rdf{rdf:XMLLiteral} for 
embedding \acro{XML} in \acro{RDF} literals. The standard further recommends 
to use datatypes from the \term{XML Schema} vocabulary
(see section \ref{sec:xmlschemas}). Other parts of the \acro{RDF} specification
\cite{Manola2004,Hayes2004,Brickley2004} provide a basic RDF vocabulary to
collect resources in classes (\rdf{rdf:type}), and resources that are used
as properties (\rdf{rdf:Property}). In addition the RDF vocabulary contains
resources to express containers (\rdf{rdf:Seq}, \rdf{rdf:Bag}, \rdf{rdf:Alt},
\rdf{rdf:_1}, \rdf{rdf:_2} \ldots), collections (\rdf{rdf:List}, 
\rdf{rdf:first}, \rdf{rdf:rest}, \rdf{rdf:nil}), primary values 
(\rdf{rdf:value}), and \Term{reification} (\rdf{rdf:Statement},
\rdf{rdf:subject}, \rdf{rdf:predicate}, \rdf{rdf:object}). Reification
is the description of \acro{RDF} triples using other \acro{RDF} triples.
This technique can be used for instance to express provenance and 
$n$-ary relationships, but it increases complexity and there are
several competing reification ontologies.\footnote{See 
\url{http://www.w3.org/TR/rdf-mt/\#Reif}, 
\url{http://www.w3.org/DesignIssues/Reify.html},
and \url{http://purl.org/ontology/prv/core\#} for other reification
ontologies.}

\subsubsection{Semantics}

A common misconception of \acro{RDF} is, that \acro{RDF} data automatically 
adheres to some semantics. The \acro{RDF} data model imposes no conditions
on the use of \acro{RDF} vocabularies to only create `meaningful' or 
`well-formed' \acro{RDF} graphs. On the contrary, an important principle 
of \acro{RDF} is that ``anyone can say anything about anything''.%
\footnote{This wording from the first \acro{RDF} concepts document
draft \cite{Klyne2004} was later modified to ``Anyone Can Make Statements About
Any Resource'' without changing the general declaration.} This means
``\acro{RDF} does not prevent anyone from making assertions that 
are nonsensical or inconsistent with other statements, or the world as 
people see it'' and ``it is not assumed that complete information about 
any resource is  available.'' \cite{Klyne2004}. The latter important 
principle is also known as \Tacro{Open World Assumption}{OWA}: the absence
of a particular statement from an \acro{RDF} graph does not mean that the 
statement is false. We illustrate this by the first triple of the graph 
in figure~\ref{fig:rdfgraphexample} and figure~\ref{fig:rdfgraphexturtle}.
The triple can be read as '`\url{info:oclcnum:318301686} is titled
`The C programming language' in English''. More precisely, it says
``the resource identified by URIref \url{info:oclcnum:318301686} has 
the English title `The C programming language', assuming the concept of 
having-a-title as identified by URIref \url{http://purl.org/dc/elements/1.1/}.''
However this statement does not imply the absence of parallel titles. The
resource may also have more than the two authors from the example --- 
we just only know that there are two authors named at least ``Kernighan'' 
and ``Ritchie'' respectively. Once we start inferencing, it could also
turn out to be one author with two names as shown in
figure~\ref{fig:rdfgraphinstances}.

The semantics that is usually associated with \acro{RDF}, does not origin 
from the \acro{RDF} data model, but from an algebra, that can be defined on 
\acro{RDF} graphs.\footnote{Some may argue, that the algebra is an
inherent part of \acro{RDF}. But this would neglect all \acro{RDF} 
applications, which do not fully implement all aspects of the \acro{RDF} 
algebra.} The algebra allows you to freely merge and 
intersect \acro{RDF} data based on simple set algebra. This is not possible 
in most other data structuring languages, for instance tree-based languages, 
which must have exactly one root element.%
\footnote{You could define union, intersection, and relative complement 
also for \acro{INI} files and some other record based data formats, but as described
in section~\ref{sec:csvandini}, these formats lack a precise definition.}
As blank nodes are always disjoint for different graphs, the simple set 
intersection of \acro{RDF} graphs cannot contain blank nodes. The same applies
to two \acro{RDF} graphs $A$ and $B$ with $A\equiv B$. The \acro{RDF} specification
uses the word `equivalent' in a different way, so we better call $A$ and $B$
`set-equivalent' or `identical' if $A\equiv B$. 

The \acro{RDF} specification defines another kind of equivalence and 
two additional relationships between \acro{RDF} graphs: \Term{graph equivalence}, 
\Term{graph instance}, and \Term{graph entailment}. Basically, two \acro{RDF} 
graphs $A$ and $B$ are \Term[equivalence (RDF)]{equivalent}, written as $A \cong B$,
if there is a bijection $M$ that maps literals to equivalent literals, URIrefs 
to equivalent URIrefs, blank nodes to blank nodes. In addition the triple
$(s,p,o)$ is in $A$ if and only if $(M(s),M(p),M(o))$ is in 
$B$ \cite[sec. 6.3]{Klyne2004}. It is recommended, but not required to 
apply Unicode Normalization before comparing graphs for equivalence.

An \acro{RDF} graph $H$ is called \Term[instance (RDF)]{instance of} an 
\acro{RDF} graph $G$, if $H$ can be obtained from $G$ by replacing zero 
or more of its blank nodes by literals, URIrefs, or other blank nodes.%
\footnote{The term `instance' is used also in
\term{RDFS} and \term{OWL} for the \rdf{rdf:type} URIref.} 
$H$ is a \Term{proper instance} of $G$, if at least one of its blank
nodes has been replaced by a non-blank node, or if at least two blank
nodes have been mapped to one. Two graphs $A$ and $B$ are equivalent 
if and only if both are instances of each other but neither a proper 
instance. Figure~\ref{fig:rdfgraphinstances} shows two possible
non-equivalent instances of the graph from figure~\ref{fig:rdfgraphexturtle}.

\begin{figure}
\begin{tabular}{ m{0.5\linewidth} m{0.5\linewidth} }
%Blank nodes replaced by URIrefs: &
%Two blank nodes merged to one: \\
\begin{lstlisting}[language=turtle]
@base <http://viaf.org/viaf/>.

<info:oclcnum:318301686> dc:title 
  "The C programming language"@en;
 dc:date "1978"^^xs:gYear;
 dc:creator <108136058>, <616522>.

<108136058> foaf:name "Kernighan".
<616522>    foaf:name "Ritchie".
\end{lstlisting}
&
\begin{lstlisting}[language=turtle]


<info:oclcnum:318301686> dc:title 
  "The C programming language"@en;
 dc:date "1978"^^xs:gYear;
 dc:creator _:b1.

 _:b1 foaf:name 
    "Kernighan", "Ritchie".
\end{lstlisting}
\\
blank nodes replaced by URIrefs &
two blank nodes merged to one
\\
\end{tabular}
\caption{Two examples of RDF graph instances}
\label{fig:rdfgraphinstances}
\end{figure}

\label{page:entailment} \Term{Entailment} is a relationship between two
\acro{RDF} graphs $A$ and $B$, that holds, if a graph equivalent to $B$ can be
created from $A$ by adding triples, based on specific inference rules. There is
not only one kind of entailment but a variety of \Term{entailment regimes} with
different sets of inference rules.\footnote{See
\url{http://www.w3.org/ns/entailment/} for a non-exclusive list of common
entailment regimes.} A specific entailment regime can be defined by an
ontology, by a set of inference rules, or by some application, that creates
entailments. If the application can handle arbitrary inference rules in some
rule language, it is also called a \Term{reasoner}. To distinguish different
entailment regimes, we say that $A$ \emph{x-entails} $B$, if $B$ is an
entailment of $A$ in regime $x$.  If some graph $G$ is not $x$-entailed by any
other graph, then $G$ can be called $x$-\Term{lean}.\footnote{The \acro{RDF}
specification only defines `lean' for simple entailment, but it can also be
defined for other entailment regimes. Finding lean graphs for a given regime is
an area of ongoing research \cite{Pichler2010}.}

The \acro{RDF} specification describes rules for simple-entailment,
rdf-entailment, and rdfs-entailment \cite[sec. 7]{Hayes2004}.
Simple entailment adds copies of existing triples by replacing
URIrefs of subject and/or object with blank nodes.\footnote{In Notation3:
\rdf{\{ ?s ?p ?o \} => \{ ?s ?p [ ] . [ ] ?p ?o \}}.} 
You can understand simple-entailment as generalization: an 
\acro{RDF} graph is always simple-entailment by all of its 
instances. \Term{rdf-entailment} extends this rule to literals. The
regime furthermore adds a rule that connect all URIrefs to 
\rdf{rdf:Property}, if they are used as property in some triple.% 
\footnote{In Notation3: \rdf{\{ ?s ?p ?o \} => \{ ?s rdf:type rdf:Property \}}.}

Entailment is an important aspect of \acro{RDF}, but it is not a feature of
\acro{RDF} data. The \acro{RDF} specification only says how to apply
entailment, but not whether and when to apply it.  In many cases inference is
expensive to calculate and would lead to a massive expansion of graphs. Some
regimes have infinite entailments also for simple graphs. Even the general
problem of determining simple-entailment between arbitrary RDF graphs is
NP-complete \cite{Hayes2004}, so most applications do not fully implement
entailment unless it is explicitly required. Testing graphs for equivalence
and instantiation is more common, but it also depends on entailment.
Entailment is also used to detect inconsistencies in \acro{RDF} data with
respect to some regime. For instance in \acro{OWL}, the triple 
\rdf{\{ ?x owl:differentFrom ?x \}} is a contradiction, that entails any 
possible triple, if \term{description logic} is applied. Reasoners for these 
entailment regimes usually detect such inconsistencies instead of infinitely 
adding triples.

The vision of the \term{Semantic Web} includes the idea of 
``intelligent agents'' that can aggregate information from distributed
sources and that can draw conclusions based on inferencing 
\cite{BernersLee2001a}. This idea requires decisions about which 
sets of \acro{RDF} data to combine, which entailment regimes to 
apply, and which URIrefs to rely on as non-ambiguous identifiers.
All these agreements are out of the scope of \acro{RDF}, which
alone is just another method of structuring data.

\begin{figure}
\centering
\begin{tikzpicture}[orm]
\entity[minimum width=1.8cm] (t) {};
\node[above=0 of t] {Triple};
\node[roles=3] (r) {};
\entity[below=6mm of r] (n) {Node};
\plays (n) to (r.one south) (n) to (r.two south) (n) to (r.three south);

\node[constraint=partition,below=of n] (nc) {} edge[inheritance] (n);

\entity[right=11.5mm of nc]  (u) {URI} edge[subtype] (nc);
\entity[below=10mm of nc] (l) {Literal} edge[subtype] (nc);
\entity[left=9.5mm of nc] (b) {Blank}  edge[subtype] (nc);

\node[roles=3,unique=1,right=6mm of l,yshift=6mm] (ld) {}; 
\node[roles=2,unique=1,right=8mm of l] (lv) {};
\node[roles=3,unique=1,right=6mm of l,yshift=-6mm] (ll) {};
\plays (ld.west) to (l) (lv.west) to (l) (ll.west) to (l);

\node[constraint=partition,below=of l.south east] (lp) {};
\limits (lp) to (ll.west) (lp) to (ld.west) (lp) to (lv.west);

\plays (ld.two north) to (u);

\value[right=10mm of lv] (s) {String};
\plays (s) to (ld.east) (s) to (lv.east) (s) to (ll.east);

%\value[right=6mm of ll.south east,yshift=-3mm] (lang) {LanguageTag};
\value[below=of ll,anchor=north west,xshift=-4mm] (lang) {LanguageTag};
\plays (ll.south) to ++(0,-4mm);

\value[right=14mm of u] (uv) {URIref};
%\draw[supinterface] (uv) to (s);
%\draw[supinterface] (lang) to (s);

\plays (uv) edge[both required] node (c) {} (u);
\binary[unique=1,unique=2] at (c) {};

%\draw[limits] (ld.one south) to node[constraint=x]{} (ll.one north);
\end{tikzpicture}
\caption{Model of generalized RDF}
\label{fig:genrdfmodel}
\end{figure}


